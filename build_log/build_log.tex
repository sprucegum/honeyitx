\documentclass{article}
\usepackage{graphicx}
\usepackage{url}
\usepackage{hevea}
\title{\textbf{"Killa B" Mini-ITX Build Log}}
\author{}
\date{\today}

\begin{document}

\maketitle

\section{Summary}
I built a very minimal PC using my 3D printer. I'm still a bit worried about damaging the fan/heatsink when I transport it, but I'm quite happy about the cooling, performance, and portability so far.
\begin{figure}[h]
\includegraphics[width=\textwidth]{pics/v1.jpg}
\emph{The first complete prototype}
\end{figure}

\section{Components}
My brother and I were talking about the possibilities of 3D printed computer cases. I saw a neat little mini-itx build\footnote{The build that inspired me: \url{https://ca.pcpartpicker.com/b/nxTBD3}} on PCPartpicker and was inspired to buy the components I needed to build this system\footnote{My part list: \url{https://ca.pcpartpicker.com/list/9Cxvq4}}.
\begin{center}
\begin{tabular}{ l l }
 CPU & AMD Ryzen 5 2400G 3.6GHz Quad-Core Processor \\  
 Motherboard & ASRock - AB350 Gaming-ITX/ac Mini ITX AM4 Motherboard \\  
 RAM & Corsair - Vengeance LPX 16GB (2 x 8GB) DDR4-3000 Memory \\
 Storage & Western Digital - Black PCIe 256GB M.2-2280 Solid State Drive \\
 PSU & ATX 300W 12VDC Power Supply + Laptop Style 90W 12VDC Power Brick \\
 Case & Custom 3D-printed  Mini-ITX case \\
\end{tabular}
\end{center}

\begin{figure}[h]
\includegraphics[width=\textwidth]{pics/nearly_bare.jpg}
\emph{Trying to visualize all the components together}
\end{figure}

I have a few reasons for building this system. I was running a dual-seat configuration before, but having two stations glued into one computer caused quite a few annoying problems. When I travel, I usually bring my laptop, but I find that my laptop doesn't have good enough game performance. Hell, my current laptop (A Dell Latitude E5450) can barely run everything I need it to run at work.
I wish I could have thrown 32GB of RAM in this machine. Unfortunately, even 16GB cost more than my processor, so I settled for 16GB. The machine is very strong and adaptable. I've run both Linux and Windows 10 Pro on it and I'm very happy with its  performance. I knew it wasn't going to be the most performant system, but I'm really hoping that the AM4 socket gets an APU that makes this machine a no-brainer upgrade. 

\section{Design}
When I was browsing Newegg, Amazon, and other sites to see what Mini-ITX cases and power supplies were available. I found that most cases are designed to accomodate a discrete GPU, Optical Drive, and 2.5" or 3.5" SATA Drives. While I may want to add a 2.5" SSD into my build in the future, I don't currently have a need for any of these extra components. They say that to a kid with a hammer, every problem looks like a nail. Well, to a nerd with a 3D Printer, I try to print a solution to every one of my problems.
\begin{figure}[h]
\includegraphics[width=\textwidth]{pics/concept.jpg}
\emph{The initial concept}
\end{figure}

I figured I could make some kind of two piece clamshell case. I saw a really cool build\footnote{The inspiration for the case: \url{https://thingiverse.com/thing:1899854}} on Thingiverse and I felt inspired. I like the aesthetics of an exposed central fan. I've heard that most mini-pc's get really hot, so cooling was a major concern.

\subsection{OpenSCAD}
I like a lot of things about OpenSCAD. Since I spend a lot of time writing code, it felt natural to also use code to draft and design physical objects. I've tried a few different CAD programs, but this is the one I've become most familiar with in the last couple years.

\begin{figure}[h]
\includegraphics[width=\textwidth]{pics/conceptSCAD.jpg}
\emph{A first attempt to describe the case in OpenSCAD}
\end{figure}
At this point, I hadn't recieved the parts, so I actually didn't know the exact measurements, let alone shape of all of my components. I started by approximating the shape, size, and position of all of the components. I tried to write the OpenSCAD code in a generic way so that I could "dial-in" the design for any motherboard and fan combo.
 
\section{Fabrication}
I printed the case using my HICTOP 3DP-12. It's a Prusa i3 clone that I purchased on sale a couple years ago. As long as you can print 180mm x 180mm, you should be able to print this. I chose to use PLA, which is the only material I really use. The case is currently held together by twist ties and hot glue.
\subsection{Iterations}
I've had to make about 3 copies of every printed part in this design. I had to eyeball a few measurements and approximate some shapes. 
\begin{figure}[h]
\includegraphics[width=\textwidth]{pics/oval_cooling.jpg}
\emph{At first I thought maybe oval cooling slots would be effective}
\end{figure}
The honeycomb design was sort of an accident. I was experimenting with shapes and placement and actually considered rectangular slats at first. I had seen some honeycomb cases and structures on Thingiverse in the past, so I figured I'd give it a shot. It was easy to make the hexagonal columns by creating a 6 sided approximation to a cylinder.

\begin{figure}[h]
\includegraphics[width=\textwidth]{pics/honeycomb_cooling.jpg}
\emph{I realized I could minimize material and optimize airflow by using a honeycomb pattern}
\end{figure}
The honeycomb pattern helped me minimize material usage while also allowing for great airflow. I had to print a handful of cases to get the fan placement and motherboard clearance just right.

\section{Assembly}

\section{Results}

\end{document}
